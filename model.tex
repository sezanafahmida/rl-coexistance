 %!TEX root = coexistence_paper.tex
\section{Background and System Model}
 
 In this section we describe the system model and the necessary background for the LPWAN technology we've considered in our design.
 
\subsection{System Model}
\revise{We consider LPWAN networks consisting of numerous nodes connected directly to one or more gateways. We consider a dense deployment where multiple LPWAN networks are operating in the same spectrum. We assume that the nodes and gateways of these networks do not posses any knowledge about each other and operate independently. Nodes are battery-powered and only wake-up if they have data to send. Nodes do not posses significant computing power and memory. Gateways are line-powered devices and are always active. We assume that the gateways can support concurrent receptions of multiple packets up to some limit. Thus, the gateways are limited by maximum number of simultaneous receptions $n$ which can be controlled by the network manager. If more than $n$ packets arrive at the gateway at the same time, some packets are dropped. Nodes rely on acknowledgements from the gateway to confirm successful reception of data. Nodes retransmit the packet if an acknowledgement is not received. The maximum number of retransmissions is also configured by the network manager.  We focus on LoRa as an representative LPWAN technologies and handle co-existence for dense LoRa networks with our q-learning framework. In the following subsection, we describe the necessary background on LoRa. }

\subsection{An Overview of LoRa}
\revise{Here we provide a brief overview of LoRa(Long-Range). Detailed description for the physical and link-layer of LoRa networks can be found in~\cite{LoRaWAN_spec_v1.1}. LoRa is the physical layer technology for an LPWAN. It's characteristics include extremely long-range which can be in the range of 3-7 miles depending on the environment and successful reception of packets even at extremely low signal-to-noise ratio. LoRa modulation is derived from \emph{Chirp Spread Spectrum (CSS)}. CSS modulation spreads the signal over the entire bandwidth, thus providing robustness to interference and enabling reception of packet at very low signal-to-noise ratio. The modulated signal consists of symbols/chirps, whose frequency linearly increases or decreases over time. Information is encoded onto each chirp using multiple cyclic shifted chips. The number of chips present in each symbol is controlled by the \emph{spreading factor}. Specifically, spreading factor is the ratio between symbol rate and chip rate. LoRa supports spreading factor in the range [6,12]. Spreading factor (SF) controls the data rate and thus the transmission time and energy consumption. A higher spreading factor reduces the data rate and thus increases the time on air for each packet leading to significant increase in energy consumption.  Transmissions on different spreading factors are orthogonal to each other. }

\revise{LoRa also supports different levels of forward error detection (FEC), called \emph{coding rates} in the range of $\frac{4}{5}$ to $\frac{4}{8}$ A higher coding rate provides resilience against bursts of interference, but increases the duration of each packet. Apart from coding rate and spreading factor, other configurable parameters for LoRa are carrier frequency, channel and bandwidth. Carrier frequency and channels vary from region to region depending on local regulation. For example, in the US band LoRa operates in the range 902-928MHz. For uplink communication in the US, there are 64 channels of bandwidth 125kHz and 8 channels of bandwidth 500kHz. For downlink, there are 8 channels of bandwidth 500kHz. }

\revise{The MAC protocol used with LoRa physical layer is called LoRa Wide Area Network(LoRaWAN). In LoRaWAN numerous nodes are directly connected to one of more gateways, which forward the data collected from nodes to a central network server. Thus, LoRa forms a star of star network topology. LoRaWAN supports classes of operation, namely class A,B and C. In all classes, nodes transmit using pure ALOHA. In class A, nodes transmit when they have data and open two receive windows after each transmission. In class B, gateway send periodic beacons to synchronize nodes. Between two beacons, the nodes wake up periodically to receive any packets from the gateway. In class C, the nodes are continuously listening for packets from the gateway.}

\revise{For the rest of the paper, we focus on LoRa as an representative LPWAN technology. We design a framework to handle coexistence in LoRa network, however, our approach can be used to handle coexistence for any LPWAN technologies easily by adjusting some parameters.}
 
